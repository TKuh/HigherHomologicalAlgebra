\documentclass[16pt,a4paper]{extarticle}
\usepackage[a4paper,margin=12mm]{geometry}
\usepackage{amsmath}
\usepackage{amsthm}
\usepackage{amssymb}
\usepackage{hyperref}
\usepackage{fancyvrb}
\usepackage{xcolor}
\usepackage{tikz}
\usepackage{tikz-cd}

\usetikzlibrary{matrix}

\theoremstyle{definition}
\newtheorem{definition}{Definition}
\newtheorem{idea}{Idea}

\theoremstyle{theorem}
\newtheorem{theorem}{Theorem}
    
\newcommand{\LL}{\mathbf{L}}
\newcommand{\HH}{\mathbf{H}}
\newcommand{\RR}{\mathbf{R}}
\newcommand{\TT}{\mathbf{T}}
\newcommand{\trunc}{\mathbf{trunc}}

\definecolor{my_color}{rgb}{0.0, 0.48, 0.65}
\definecolor{rule_color}{rgb}{1.0, 0.65, 0.0}

    \title{Example about BGG correspondence}
    \author{Kamal Saleh}
    \begin{document}
    \maketitle
    \tableofcontents
    \section{The functors $\RR$ and $\LL$}
    In the following $S$ is the graded polynomial ring $\mathbb{Q}[x_0,\dots,x_n]$ with 
    $\mathrm{deg}(x_i)=1,i=0,\dots,n$ and $A$ is its dual graded ring, i.e., the exterior algebra
    generated by $e_i,i=0,\dots,n$ with $\mathrm{deg}(e_i)=-1$ and $\omega_A := \mathrm{Hom}_k(E,k)\cong A(n+1)$.
    \begin{definition}
        Given a graded $S$-module $M=\bigoplus_{d\in\mathbb{Z}} M_d$, we construct the following cochain complex of graded $A$-modules:
        $$\RR(M):\dots \rightarrow M_{i-1}\otimes_k \omega_A\rightarrow M_{i}\otimes_k \omega_A \rightarrow \dots$$
        where the term $M_{i}\otimes_k \omega_A$ has cohomological degree $i$.
    \end{definition}
    \begin{definition}
        Given a graded $A$-module $P=\bigoplus_{d\in\mathbb{Z}} P_d$, we construct the following cochain complex of graded $S$-modules:
        $$\LL(P):\dots \rightarrow S \otimes_k P_{j}\rightarrow S \otimes_k P_{j-1} \rightarrow \dots$$
        where the term  $S \otimes_k P_{j}$ has cohomological degree $-j$.
    \end{definition}
    \begin{theorem}
        If $M$ is finitely generated graded $S$-module and $P$ is a finitely generated graded $A$-module, then $\LL(P)$ is free resolution of $M$ if and 
        only if $\RR(M)$ is an injective resolution of $P$.
    \end{theorem}
    \begin{idea}
        If $M$ is finitely generated graded $S$-module and $r\geq\mathrm{reg}(M)$, then $\LL(\HH^r( \RR^{> r-1}(M)) )$ is free resolution of $M_{\geq r}$.\\
        Here $\RR^{> r-1}(M) = \trunc^{> r-1}_{below}(\RR(M))$.
        Of course, we can replace $\RR$ by $\TT$ (the Tate functor).
    \end{idea}

\begin{Verbatim}[commandchars=\\\{\}, fontseries=b, frame=single, label=Gap Code, framerule=0.5mm, rulecolor=\color{rule_color}]
    \textcolor{my_color}{S;}
    \textcolor{my_color}{A;}
    \textcolor{my_color}{m := RandomMatrixBetweenGradedFreeLeftModules( [ 5, 4 ],[ 4, 2, 3, 1 ], S );}
    \textcolor{my_color}{M := AsGradedLeftPresentation( m, [ 4, 2, 3, 1 ] );}
    \textcolor{my_color}{Display( M );}
    \textcolor{my_color}{r := Maximum( 1, CastelnuovoMumfordRegularity( M ) ) + 1;}
    \textcolor{my_color}{M_geq_r := GradedLeftPresentationGeneratedByHomogeneousPart( M, r );}
    \textcolor{my_color}{R := RFunctor( S );}
    \textcolor{my_color}{trunc_g_rm1_below := BrutalTruncationBelowFunctor( cochains_graded_lp_cat_ext, r - 1 );}
    \textcolor{my_color}{H_r := CohomologyFunctorAt( cochains_graded_lp_cat_ext, graded_lp_cat_ext, r );}
    \textcolor{my_color}{L := LFunctor( S );}
    \textcolor{my_color}{Free_res := PreCompose( [ R, trunc_g_rm1_below, H_r, L ] );}
    \textcolor{my_color}{F := ApplyFunctor( Free_res, M_geq_r );}
    \textcolor{my_color}{RM_geq_r := ApplyFunctor(R,M_geq_r);;}
    \textcolor{my_color}{P := Source( CyclesAt( RM_geq_r, r ) );;}
    \textcolor{my_color}{P_leq_r := GradedLeftPresentationGeneratedByHomogeneousPart(P,r);;}
    \textcolor{my_color}{emb_P_leq_r_in_P := EmbeddingInSuperObject( P_leq_r );}
    \textcolor{my_color}{h := PreCompose( emb_P_leq_r_in_P, CyclesAt( RM_geq_r, r ) );}
    \textcolor{my_color}{mat := UnderlyingMatrix(h);}
    \textcolor{my_color}{mat := DecompositionOfHomalgMat(mat)[2^(l+1)][2]*S;}
    \textcolor{my_color}{t := GradedPresentationMorphism( F[ -r ], mat, M_geq_r );}
    \textcolor{my_color}{IsZero( PreCompose( F^(-r-1), t ) );}
    \textcolor{my_color}{iso := CokernelColift( F^(-r-1), t );}
    \textcolor{my_color}{IsIsomorphism( iso );}
\end{Verbatim}
    
\begin{idea}
    If $M$ is finitely generated graded $S$-module and $r\geq\mathrm{reg}(M)$, Then the exactness of $\TT(M)$ implies
    $$\HH^{r-1}( \TT^{\leq r-1}(M)) \cong \HH^r( \TT^{> r-1}(M)),$$ hence, 
    $$\LL(\HH^{r-1}( \TT^{\leq r-1}(M)) ) \cong  \LL(\HH^r( \TT^{> r-1}(M)) )$$ are isomorphic. In particular 
    $\LL(\HH^{r-1}( \TT^{\leq r-1}(M)) )$ is free resolution of $M_{\geq r}$.
    Here $\TT^{\leq r-1}(M) = \trunc^{\leq r-1}_{above}(\TT(M))$.

    This isomorphism can be simply computed by applying the functor $H^{r-1}$ on the 
    natural cochain morphism 
    $$\psi: \TT^{\leq r-1}(M)\rightarrow \TT^{> r-1}(M)[1]^{\mathrm{unsigned}}.$$
    
\end{idea}

\begin{Verbatim}[commandchars=\\\{\}, fontseries=b, frame=single, label=Gap Code, framerule=0.5mm, rulecolor=\color{rule_color} ]
    \textcolor{my_color}{m := RandomMatrixBetweenGradedFreeLeftModules( [ 5, 4 ],[ 4, 2, 3, 1 ], S );;}
    \textcolor{my_color}{M := AsGradedLeftPresentation( m, [ 4, 2, 3, 1 ] );;}
    \textcolor{my_color}{r := Maximum( 1, CastelnuovoMumfordRegularity( M ) ) + 1;;}
    \textcolor{my_color}{M_geq_r := GradedLeftPresentationGeneratedByHomogeneousPart( M, r );;}
    \textcolor{my_color}{Display( M_geq_r );;}
    \textcolor{my_color}{T := TateFunctor(S);;}
    \textcolor{my_color}{trunc_leq_rm1 := BrutalTruncationAboveFunctor( cochains_graded_lp_cat_ext, r-1 );;}
    \textcolor{my_color}{trunc_g_rm1 := BrutalTruncationBelowFunctor( cochains_graded_lp_cat_ext, r-1 );;}
    \textcolor{my_color}{unsigned_shift_by_1 := UnsignedShiftFunctor( cochains_graded_lp_cat_ext, 1 );;}
    \textcolor{my_color}{coh_rm1 := CohomologyFunctorAt( cochains_graded_lp_cat_ext, graded_lp_cat_ext, r-1 );;}
    \textcolor{my_color}{psi := CochainMorphism(}
            \textcolor{my_color}{ApplyFunctor( PreCompose([T,trunc_leq_rm1]), M_geq_r ),}
            \textcolor{my_color}{ApplyFunctor( PreCompose([T,trunc_g_rm1, unsigned_shift_by_1]), M_geq_r ),}
            \textcolor{my_color}{[ ApplyFunctor(T,M_geq_r)^(r-1) ],}
            \textcolor{my_color}{r-1 );}
    \textcolor{my_color}{iso := ApplyFunctor( coh_rm1, psi );;}
    \textcolor{my_color}{IsIsomorphism(iso);}
\end{Verbatim}

    
    \begin{idea}
        If $M$ is finitely generated graded $S$-module and $r\geq\mathrm{reg}(M)$.
    \end{idea}
    \begin{Verbatim}[commandchars=\\\{\}, fontseries=b, frame=single, label=Gap Code, framerule=0.5mm, rulecolor=\color{rule_color} ]
    \textcolor{my_color}{m := RandomMatrixBetweenGradedFreeLeftModules( [ 5, 4 ],[ 4, 2, 3, 1 ], S );;}
    \textcolor{my_color}{M := AsGradedLeftPresentation( m, [ 4, 2, 3, 1 ] );;}
    \textcolor{my_color}{r := Maximum( 1, CastelnuovoMumfordRegularity( M ) )+1;;}
    \textcolor{my_color}{M_geq_r := GradedLeftPresentationGeneratedByHomogeneousPart( M, r );;}
    \textcolor{my_color}{trunc_leq_rm1 := BrutalTruncationAboveFunctor( cochains_graded_lp_cat_ext, r-1 );;}
    \textcolor{my_color}{T := TateFunctor(S);;}
    \textcolor{my_color}{trunc_leq_m1 := BrutalTruncationAboveFunctor( cochains_graded_lp_cat_sym, -1 );;}
    \textcolor{my_color}{ch_trunc_leq_m1 := ExtendFunctorToCochainComplexCategoryFunctor(trunc_leq_m1 );;}
    \textcolor{my_color}{complexes_sym := CochainComplexCategory( cochains_graded_lp_cat_sym );;}
    \textcolor{my_color}{bicomplxes_sym := AsCategoryOfBicomplexes(complexes_sym);;}
    \textcolor{my_color}{complexes_to_bicomplex := ComplexOfComplexesToBicomplexFunctor(complexes_sym, bicomplxes_sym );;}
    \textcolor{my_color}{L := LFunctor(S);;}
    \textcolor{my_color}{chL := ExtendFunctorToCochainComplexCategoryFunctor(L);;}
    \textcolor{my_color}{trunc_leq_rm1_TM_geq_r := ApplyFunctor( PreCompose(T,trunc_leq_rm1), M_geq_r );;}
    \textcolor{my_color}{phi := CochainMorphism(}
        \textcolor{my_color}{trunc_leq_rm1_TM_geq_r,}
        \textcolor{my_color}{StalkCochainComplex( CokernelObject( trunc_leq_rm1_TM_geq_r^(r-2) ), r-1 ),}
        \textcolor{my_color}{[ CokernelProjection( trunc_leq_rm1_TM_geq_r^(r-2) ) ],}
        \textcolor{my_color}{r-1 );}
    \textcolor{my_color}{IsWellDefined( phi,2,4);;}
    \textcolor{my_color}{mor := ApplyFunctor( PreCompose( [ chL, ch_trunc_leq_m1, complexes_to_bicomplex ] ), phi );;}
    \textcolor{my_color}{tau := ComplexMorphismOfHorizontalCohomologiesAt(mor,r-1);;}
    \end{Verbatim}

\begin{idea}
    In the previous right and above bounded bicomplex, the Beilinson Monad is the cochain complex of the vertical cohomologies at the line 
    with cohomological index $-1$.
\end{idea}
\begin{tikzpicture}[baseline= (a).base]
    \node[scale=0.8] (a) at (0,0){
\begin{tikzcd}
 &  & i & i+1 & \cdots & 0 & \cdots & r-1 & r \\    
0 &  & 0 & 0 & \cdots & 0 &  & 0 &  \\
-1 & \cdots \arrow[r] & S\otimes_k T_1^i \arrow[r] \arrow[u] & S\otimes_k T_1^{i+1} \arrow[r] \arrow[u] & \cdots \arrow[r] \arrow[u] & S\otimes_k T_1^{0} \arrow[u] \arrow[r] & \cdots \arrow[r] & S\otimes_k T_1^{r-1}(=0) \arrow[u] \arrow[r] & 0 \\
 &  \arrow[r] & \vdots \arrow[u] \arrow[r] & \vdots \arrow[u] \arrow[r] & \cdots \arrow[r] \arrow[u] & \vdots \arrow[u] \arrow[r] & \cdots \arrow[r] & \cdots(=0) \arrow[u] &  \\
-r+1 &  \arrow[r] & S\otimes_k T^i_{r-1} \arrow[u] \arrow[r] & S\otimes_k T^{i+1}_{r-1} \arrow[u] \arrow[r] & \cdots \arrow[r] \arrow[u] & S\otimes_k T_{r-1}^{0} \arrow[u] \arrow[r] & \cdots \arrow[r] & S\otimes_k T^{r-1}_{r-1} \arrow[u] \arrow[r] & 0 \\
-r &  \arrow[r] & S\otimes_k T^i_{r} \arrow[u] \arrow[r] & S\otimes_k T^{i+1}_{r} \arrow[u] \arrow[r] & \cdots \arrow[r] \arrow[u] & S\otimes_k T_{r}^{0} \arrow[u] \arrow[r] & \cdots \arrow[r] & S\otimes_k T_{r}^{r-1} \arrow[u] \arrow[r] & 0 \\
% &  &  \arrow[u] &  \arrow[u] &  \arrow[u] &  \arrow[u] &  &  \arrow[u] & \\
 &  & \vdots & \vdots& \vdots & \vdots &  & \vdots \arrow[u] &  \\ 
 -r-n&  & \vdots & \vdots& \vdots & \vdots &  & S\otimes_k T^{r-1}_{n+r}\arrow[r]\arrow[u] & 0 \\ 
\end{tikzcd}
   };
\end{tikzpicture}
\end{document}